\chapter{Teakwood System }

\section{Overview}
Structurally, like most websites, Teakwood system has a three layers layout: frontend, backend, and database. Becuase Teakwood uses computing servers to run jobs, so the eco Teakwood system actually has four layers; the up mentioned three plus a computing layer. the below figure shows how is the four-layer structure looks like:\\

\begin{figure}[htb]
\centering
\includegraphics[scale=0.5]{./system_structure} 
% e.g. insert ./image for image.png in the working directory, adjust scale as necessary
\caption{Teakwood System Overview}
\label{fig:label} % insert suitable label, this is used to refer to a fig from within the text as shown above
\end{figure}

On the above figure, we can see the four straight layers lays bottom-up. Each layer from left to right: the left part is the layer name; the middle part is the layer content; and the right part is the layer reality. Let's go through these layers one by one.

\section{Frontend}
The frontend is a visible GUI that the user interact with. Basically all what we can see in the Teakwood website can be called the "frontend".  For neat purpose, Teakwood separated the frontend into two parts: the \textbf{information pages} and the \textbf{working console}. See below:

\begin{figure}[htb]
\centering
%\includegraphics[scale=0.5]{./system_structure} 
\includegraphics[scale=0.6]{./website_structure} % e.g. insert ./image for image.png in the working directory, adjust scale as necessary
\caption{Website Strucutre}
\label{fig:label} % insert suitable label, this is used to refer to a fig from within the text as shown above
\end{figure}

The \textbf{information pages} introduces you all general things about Teakwood. For examples:\\

$\bullet$ What is Teakwood?\\
$\bullet$ What can Teakwood do?\\
$\bullet$ How to install Teakwood?\\
$\bullet$ How to use Teakwood?\\
$\bullet$ The user manual.\\
$\bullet$ Video tutorial.\\
$\bullet$ Teakwood forum.\\


The \textbf{working console} is the working place you actually tango with your jobs and data. The functional buttons will guide you to different functional web pages for different purpose.\\
Let me explain these functional buttons one by one.\\

$\bullet$ \textbf{Home}: Directs you to the homepage.\\
$\bullet$ \textbf{Documents}: Directs you to the user manual.\\
$\bullet$ \textbf{About}: Teakood self introduction.\\
$\bullet$ \textbf{User}: Displays user information.\\
$\bullet$ \textbf{Projects}: Creates projects and provies project overview.\\
$\bullet$ \textbf{Models}: Create models and provides models overview.\\
$\bullet$ \textbf{Jobs}:Creates jobs, overview jobs and job monitoring.\\
$\bullet$ \textbf{Report}:Download job output.\\
$\bullet$ \textbf{Management}:Access to the admin system.\\


Note the color differences in the bottom layer.\\

$\bullet$ The \textbf{Green}: all visitors can see and manipulate.\\
$\bullet$ The \textbf{orange}: only logging user can see and manipulate.\\
$\bullet$ The \textbf{red}: only superuser can see and manipulate.\\


\section{Backend}
The backend is the logical design on how to interact with user. This Includes verifying user's request, pulling requested data, generating HTML web page and displaying web page. Teakwood follows the MVC(Model-View-Controller)design pattern and separates all the functions in to loose coupling parts, in Django, it call "app". All parts can both work independently and cooperatively.(We will have a backend chapter to reveal the logic mystery.)

In Teakwood system, there are mainly eight parts(Apps).\\

$\bullet$ \textbf{Teakwood}: control the frontend presentation.\\
$\bullet$ \textbf{Accounts}: control the user identification.\\
$\bullet$ \textbf{Models}: invoke and control the computing models.\\
$\bullet$ \textbf{machine}:invoke and control the computing resources\\
$\bullet$ \textbf{PBS}:Guide the PBS script generation.\\
$\bullet$ \textbf{File upload}:gather input files to buffer for downloading \\
$\bullet$ \textbf{Admin}:overall control of user and data.\\
$\bullet$ \textbf{Celery}:asynchronous handling.\\

\section{Data handling}
Teakwood system handles three types of data: the website data, the computing data and the message queue data. For each type of data we provide a different storage. see this table:\\
\\   
\begin{table}[h]
\begin{tabular}{lllll}
\cline{1-3}
\multicolumn{1}{|c|}{\textbf{Website data}} & \multicolumn{1}{c|}{\textbf{Computing data}} & \multicolumn{1}{c|}{\textbf{Message queue data}} &  &  \\ \cline{1-3}
\multicolumn{1}{|c|}{\textbf{MySQL}} & \multicolumn{1}{c|}{\textbf{File server}} & \multicolumn{1}{c|}{\textbf{redis server}} &  &  \\ \cline{1-3}
\multicolumn{1}{|c|}{\textbf{Teakwood data}} & \multicolumn{1}{c|}{\textbf{inputs and outputs}} & \multicolumn{1}{c|}{\textbf{Asynchronous handling}} &  &  \\ \cline{1-3}
                                &                                &                                &  & 
\end{tabular}
\end{table}


Teakwood website uses MySQL database for store it website data, e.g. the user account and the project labels.\\
For the computing data, we periodically rsync them to a separate file server for data  backup and downloading purpose. e.g. input files and the output result.\\
Message queue data is generated when we use Celery to asynchronous processing time consuming process. They are just ephemeral data, so we simply use a redis server to keep it.\\

\section{Remote Configuration}
Before the first time we can run a job in HPC or cloud, we have set up a connection and ready everything. the main things we should done are:\\

$\bullet$ Establish an password-less ssh log-in.\\
$\bullet$ Compile the tools and packages we will use in remote machine.\\
$\bullet$ Ready all the import path for Teakwood to use.\\

One those steps are done, we just simply "plug-in" Teakwood to the remote machine, and everything we can do from Teakwood web portal, without touch the under layers.


%\subsection{<Sub-section title>}

%\subsection{<Sub-section title>}
%some text\cite{citation-2-name-here}, some more text

%Refer figure \ref{fig:label}.


