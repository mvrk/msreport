\begin{center}
\textbf{Preface}\\[1.0cm]
\end{center}

Four years ago, I know nothing about Linux. I then got a job to deal with HPC(High Performance Computing) systems. Suddenly, I jumped into an Linux environment. I found that all those GUIs(Graphic User Interfaces), which I used to when using Windows OS(Operating System), are gone, and I have to use Linux commands to manipulate those machines and manage my stuff. This makes me uncomfortable and cumbersome. With time goes by, I eventually can work with Linux comfortably. However, I still have the tendency to use GUI when proving me the option.\\
I found that I am not the only one that had such experience. In my working building, a lot of scientists and researchers are suffering this pain. Not only this, since these researchers usually have to work on cross-platforms, the OS conflict add them many wasteful jobs. For example, scientist A want to share a computing result to researcher B, as B needs it to do the visualization. Let's assume A uses Linux and B uses Windows. What should they do? Normally without tool's help, then need to type commands. Command, command, command... copy, paste, and then command, command ...\\Why not create a GUI platform to reduce those repeated typing work and allow them to work cooperatively in one place? With this desire, Here comes the Teakwood framework.




